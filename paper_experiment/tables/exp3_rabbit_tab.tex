
\documentclass[12pt]{article}


\begin{document}
\begin{table}
    \caption{Regression results for count-the-rabbits tasks}
    \vspace*{12pt}
    \centering

      \begin{tabular}{lllll}
\hline
 & (1) Pooled & (2) FE & (3) FE & (4) FE \\
\hline
Group & -0.732$^{*}$ & -4.466$^{*}$ & -0.21 & -0.964$^{***}$ \\
 & (0.331) & (1.789) & (1.771) & (0.312) \\
Question$\cdot1\{\text{Group}=0\}$ & 0.581$^{*}$ & 1.357$^{*}$ & 1.618$^{*}$ & 2.032$^{**}$ \\
 & (0.228) & (0.606) & (0.681) & (0.778) \\
Questsion$\cdot1\{\text{Group}=1\}$ & 0.07 & 0.225 & 0.466 & 0.214 \\
 & (0.319) & (0.376) & (0.444) & (0.352) \\\hline

observations & 3504 & 3504 & 2190 & 810 \\
AIC & 883.789 & 1103.793 & 752.785 & 586.582 \\
\hline
\end{tabular}
% INSERT reg_rabbit

    \vspace*{4pt}
    \centering
    \begin{minipage}{0.85\textwidth}
    {\par\footnotesize Note: * $p<0.05$, ** $p<0.01$, *** $p<0.005$. Standard errors are clustered at the subject level and are reported in the parentheses. The $p$-values are calculated based on Wald tests. FE indicates a fixed-effect model (containing participant-specific dummies). Column (1)-(2) are estimated on the full sample, Column (3) contains the participants who have changed choices at least once in intertemporal choice tasks, Column (4) contains those having made at least one wrong choice in count-the-rabbits tasks. Each model includes task-specific dummies as intercepts. }
    \end{minipage}
    \label{tab:exp3_reg_rabbit_choice}
\end{table}

\end{document}

