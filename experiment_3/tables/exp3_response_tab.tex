
\documentclass[12pt]{article}


\begin{document}
\begin{table}
    \caption{The Relationship Between Response Time, Choice and Experimental Conditions}
    \vspace*{12pt}
    \centering

      \begin{tabular}{lll}
\hline
 & Intertemporal Choice & Rabbit \\
\hline
Group & -0.684$^{***}$ & -0.792$^{***}$ \\
 & (0.141) & (0.144) \\
Question$\cdot1\{\text{Group}=0\}$ & -0.165 & 0.912$^{***}$ \\
 & (0.174) & (0.199) \\
Question$\cdot1\{\text{Group}=1\}$ & 0.457$^{***}$ & 0.849$^{***}$ \\
 & (0.101) & (0.132) \\
Choice & 0.954$^{*}$ & 1.291$^{***}$ \\
 & (0.399) & (0.456) \\
Choice$\times$Group & -0.762$^{*}$ & -1.265$^{***}$ \\
 & (0.304) & (0.229) \\
Choice$\times$Question$\cdot1\{\text{Group}=0\}$ & 0.001 & -0.138 \\
 & (0.257) & (0.23) \\
Choice$\times$Question$\cdot1\{\text{Group}=1\}$ & -0.12 & 0.263 \\
 & (0.195) & (0.175) \\\hline

observations & 4393 & 2179 \\
aic & 18560.034 & 8711.872 \\
adj-$R^2$ & 0.381 & 0.55 \\
\hline
\end{tabular}
% INSERT reg_response_time

    \vspace*{4pt}
    \centering
    \begin{minipage}{0.85\textwidth}
    {\par\footnotesize Note: * $p<0.05$, ** $p<0.01$, *** $p<0.005$. Both models are estimated through 2SLS method. Standard errors are clustered at the subject level and are reported in the parentheses. The p-values are calculated based on t-tests. Indepdent variables for each second-stage regression include an intercept, task-specific dummies and their interactions with the predicted choice, and subject-specific dummies.}
    \end{minipage}
    \label{tab:exp3_reg_response_time}
\end{table}

\end{document}

