
\documentclass[12pt]{article}


\begin{document}
\begin{table}
    \caption{Regression Results for Intertemporal Choice Tasks}
    \vspace*{12pt}
    \centering

      \begin{tabular}{llll}
\hline
 & (1) Pooled & (2) FE & (3) FE \\
\hline
Group & 0.085 & 0.047 & -0.096 \\
 & (0.19) & (0.533) & (0.154) \\
Question$\cdot1\{\text{Group}=0\}$ & 0.085 & 0.301 & 0.301 \\
 & (0.059) & (0.189) & (0.187) \\
Question$\cdot1\{\text{Group}=1\}$ & 0.218$^{***}$ & 0.53$^{***}$ & 0.53$^{***}$ \\
 & (0.067) & (0.164) & (0.163) \\\hline

observations & 7056 & 4416 & 7056 \\
aic & 9625.349 & 4253.529 & 4259.531 \\
\hline
\end{tabular}
% INSERT reg_choice

    \vspace*{4pt}
    \centering
    \begin{minipage}{0.85\textwidth}
    {\par\footnotesize Note: * $p<0.05$, ** $p<0.01$, *** $p<0.005$. Standard errors are clustered at the subject level and are reported in the parentheses. The p-values are calculated based on Wald tests. Model (2) includes subject-specific dummies for those having changed their choices at least once across all intertemporal choice tasks. Model (3) adds two dummies to Model (2) to capture whether a subject always chooses the sequence option, and the single option. Each model includes an intercept and task-sepcific dummies.}
    \end{minipage}
    \label{tab:exp3_reg_intertemporal_choice}
\end{table}

\end{document}

