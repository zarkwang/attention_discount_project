% Options for packages loaded elsewhere
\PassOptionsToPackage{unicode}{hyperref}
\PassOptionsToPackage{hyphens}{url}
%
\documentclass[
  12pt,
]{article}
\usepackage{amsmath,amssymb}
\usepackage{iftex}
\ifPDFTeX
  \usepackage[T1]{fontenc}
  \usepackage[utf8]{inputenc}
  \usepackage{textcomp} % provide euro and other symbols
\else % if luatex or xetex
  \usepackage{unicode-math} % this also loads fontspec
  \defaultfontfeatures{Scale=MatchLowercase}
  \defaultfontfeatures[\rmfamily]{Ligatures=TeX,Scale=1}
\fi
\usepackage{lmodern}
\ifPDFTeX\else
  % xetex/luatex font selection
\fi
% Use upquote if available, for straight quotes in verbatim environments
\IfFileExists{upquote.sty}{\usepackage{upquote}}{}
\IfFileExists{microtype.sty}{% use microtype if available
  \usepackage[]{microtype}
  \UseMicrotypeSet[protrusion]{basicmath} % disable protrusion for tt fonts
}{}
\makeatletter
\@ifundefined{KOMAClassName}{% if non-KOMA class
  \IfFileExists{parskip.sty}{%
    \usepackage{parskip}
  }{% else
    \setlength{\parindent}{0pt}
    \setlength{\parskip}{6pt plus 2pt minus 1pt}}
}{% if KOMA class
  \KOMAoptions{parskip=half}}
\makeatother
\usepackage{xcolor}
\usepackage[margin=1in]{geometry}
\usepackage{graphicx}
\makeatletter
\def\maxwidth{\ifdim\Gin@nat@width>\linewidth\linewidth\else\Gin@nat@width\fi}
\def\maxheight{\ifdim\Gin@nat@height>\textheight\textheight\else\Gin@nat@height\fi}
\makeatother
% Scale images if necessary, so that they will not overflow the page
% margins by default, and it is still possible to overwrite the defaults
% using explicit options in \includegraphics[width, height, ...]{}
\setkeys{Gin}{width=\maxwidth,height=\maxheight,keepaspectratio}
% Set default figure placement to htbp
\makeatletter
\def\fps@figure{htbp}
\makeatother
\setlength{\emergencystretch}{3em} % prevent overfull lines
\providecommand{\tightlist}{%
  \setlength{\itemsep}{0pt}\setlength{\parskip}{0pt}}
\setcounter{secnumdepth}{-\maxdimen} % remove section numbering
\usepackage{setspace} \usepackage{amsmath} \usepackage{array} \usepackage{caption} \usepackage{longtable} \usepackage{booktabs} \usepackage{enumitem} \renewcommand{\arraystretch}{1} \captionsetup[table]{skip=5pt} \setstretch{1.5}
\ifLuaTeX
  \usepackage{selnolig}  % disable illegal ligatures
\fi
\usepackage[]{natbib}
\bibliographystyle{apalike}
\IfFileExists{bookmark.sty}{\usepackage{bookmark}}{\usepackage{hyperref}}
\IfFileExists{xurl.sty}{\usepackage{xurl}}{} % add URL line breaks if available
\urlstyle{same}
\hypersetup{
  pdftitle={Proof},
  pdfauthor={Zark Zijian Wang},
  hidelinks,
  pdfcreator={LaTeX via pandoc}}

\title{Proof}
\author{Zark Zijian Wang}
<<<<<<< HEAD
\date{January 05, 2024}
=======
\date{December 02, 2023}
>>>>>>> 06181f9ba6a6863cc3f71040b7af51c1e271a799

\begin{document}
\maketitle

<<<<<<< HEAD
Optimal discounting

\[
\begin{aligned}
&\max_{\mathcal{W}}\;&&\sum_{t=0}^T w_tu(s_t) - C(\mathcal{W}) \\
&s.t.\; &&\sum_{t=0}^Tw_t =1 \\
&&& w_t >0 \text{ for all } t\in \{0,1,...,T\}
\end{aligned}
\]

separable information cost function

\[
C(\mathcal{W})=\sum_{t=0}^Tf_t(w_t)
\]

Axiom 1 (sequential outcome betweenness) For any \(s_{0\rightarrow T}\),
there exists a \(\alpha\in(0,1)\) such that
\(s_{0\rightarrow T} \sim \alpha\cdot s_{0\rightarrow T-1}+(1-\alpha) \cdot s_T\).

Axiom 2 (sequential bracket independence) For any
=======
We define \(s_{0\rightarrow T}=[s_0,s_1,...,s_T]\) as a sequence of
rewards, starting at period 0 and ending at period \(T\). Similarly,
\(s_{0\rightarrow t}=[s_0,…,s_t]\) is defined as a sub-sequence of it.
Let \(\mathcal{W}=[w_0,...,w_T]\) denote the attention weights for all
rewards in sequence \(s_{0\rightarrow T}\), where \(W\in[0,1]^{T+1}\).
Let \(C(\mathcal{W})\) denote the information cost function. We can
construct the following constrained optimal discounting problem for
\(s_{0\rightarrow T}\):\[\tag{A.1}
\begin{aligned}
\max_{\mathcal{W}}  \quad &\sum_{t=0}^T w_tu(s_t) - C(\mathcal{W}) \\
s.t. \quad &  \sum_{t=0}^T w_t=1,\; w_t \geq 0 \text{ for all } t\in\{0,1,…,T\} \\ \\
\end{aligned}
\]We assume \(C(\mathcal{W})\) is constituted by separable costs, that
is, \(C(\mathcal{W})=\sum_{t=0}^Tf_t(w_t)\), where \(f_t(w_t)\) is an
increasing and convex function.

Axiom 1 (sequential outcome-betweenness) For any \(s_{0\rightarrow T}\),
there exists a \(\alpha\in(0,1)\) such that
\(s_{0\rightarrow T} \sim \alpha\cdot s_{0\rightarrow T-1}+(1-\alpha) \cdot s_T\).

Axiom 2 (sequential bracket-independence) For any
>>>>>>> 06181f9ba6a6863cc3f71040b7af51c1e271a799
\(s_{0\rightarrow T}\), if there exists non-negative real numbers
\(\alpha_1\), \(\alpha_2\), \(\beta_0\), \(\beta_1\), \(\beta_2\), such
that
\(s_{0\rightarrow T}\sim \alpha_1 \cdot s_{0\rightarrow T-1} + \alpha_2 \cdot s_{T}\),
and
\(s_{0\rightarrow T}\sim \beta_0 \cdot s_{0\rightarrow T-2}+\beta_1 \cdot s_{T-1}+\beta_2 \cdot s_{T}\),
then we must have \(\alpha_2 = \beta_2\).

<<<<<<< HEAD
Axiom 3 (state independence) \(s_t \succ s'_t\) implies that for any
\(\alpha \in (0,1)\) and reward \(c\),
\(\alpha \cdot s_t + (1-\alpha)\cdot c \succ \alpha \cdot s'_t + (1-\alpha)\cdot c\).

Axiom 4 (aggregate invariance to constant sequences) Consider two
constant sequences, denoted as \(s_c\) and \(s'_c\), where each element
in \(s_c\) equals \(c\) and each element in \(s'_c\) equals \(c'\). For
any \(s_{0\rightarrow T}\), \(s'_{0\rightarrow T}\) and
=======
Axiom 3 (aggregate invariance to constant sequences) Consider two
constant sequences, denoted as \(s_c\) and \(s'_c\), where each element
in \(s_c\) is equal to \(c\) and each element in \(s'_c\) is equal to
\(c'\). For any \(s_{0\rightarrow T}\), \(s'_{0\rightarrow T}\) and
>>>>>>> 06181f9ba6a6863cc3f71040b7af51c1e271a799
\(\alpha\in(0,1)\), if
\(\alpha \cdot s_t+(1-\alpha)\cdot c\sim\alpha \cdot s'_t+(1-\alpha)\cdot c'\)
holds for every \(t\), then
\(\alpha \cdot s_{0\rightarrow T}+(1-\alpha)\cdot s_c\sim \alpha \cdot s'_{0\rightarrow T}+(1-\alpha)\cdot s'_c\).

<<<<<<< HEAD
=======
Axiom 4 (state independence) If \(s_t \succ s'_t\), then for any
\(\alpha \in (0,1)\) and reward \(c\),
\(\alpha \cdot s_t + (1-\alpha)\cdot c \succ \alpha \cdot s'_t + (1-\alpha)\cdot c\).

Proposition: \(\succsim\) admits an ADU representation if it admits a DU
representation subject to the constrained optimal discounting problem,
and satisfies Axiom 1-4.

>>>>>>> 06181f9ba6a6863cc3f71040b7af51c1e271a799
\textbf{Proof.}

\emph{Lemma 1}. If Axiom 1 holds, for any \(s_{0\rightarrow T}\), there
exist non-negative real numbers \(w_0\), \(w_1\),\ldots, \(w_T\) such
that
\(s_{0\rightarrow T} \sim w_0 \cdot s_0 +w_1\cdot s_1 + ...+w_T\cdot s_T\)
where \(\sum_{t=0}^T w_t=1\).

<<<<<<< HEAD
When \(T=1\), the lemma is a direct application of Axiom 1.

When \(T\geq 2\), according to Axiom 1, for any \(2\leq t\leq T\), there
should exist a real number \(\alpha_t\in(0,1)\) such that
\(s_{0\rightarrow t}\sim \alpha_t\cdot s_{0\rightarrow t-1}+(1-\alpha_t)\cdot s_{t}\).
For sequence \(s_{0\rightarrow T}\), we can recursively apply such
=======
When \(T=1\), the claim of Lemma 1 is a direct application of Axiom 1.
When \(T\geq 2\), according to Axiom 1, for any \(2\leq t\leq T\), there
should exist a real number \(\alpha_t\in(0,1)\) such that
\(s_{0\rightarrow t}\sim \alpha_t\cdot s_{0\rightarrow t-1}+(1-\alpha_t)\cdot s_{t}\).
For sequence \(s_{0\rightarrow T}\), we can recursively apply these
>>>>>>> 06181f9ba6a6863cc3f71040b7af51c1e271a799
preference relations as follows:

\[
\begin{aligned}
s_{0\rightarrow T} &\sim \alpha_{T-1}\cdot s_{0\rightarrow T-1} + (1-\alpha_{T-1})\cdot s_T \\
&\sim  \alpha_{T-1}\alpha_{T-2}\cdot s_{0\rightarrow T-2} + \alpha_{T-1}(1-\alpha_{T-2})\cdot s_{T-1} + (1-\alpha_{T-1})\cdot s_T \\
& \sim ...\\
& \sim w_0 \cdot s_0 + w_1\cdot s_1 +... +w_T\cdot s_T
\end{aligned}
\]

where \(w_0=\prod_{t=0}^{T-1}\alpha_t\), \(w_T = 1-\alpha_{T-1}\), and
for \(0<t<T\),
\(w_t=(1-\alpha_{t-1})\prod_{\tau=t}^{T-1}\alpha_{\tau}\). It is easy to
show the sum of all these weights, denoted by \(w_t\)
(\(0\leq t\leq T\)), equals 1.

<<<<<<< HEAD
Therefore, if Axiom 1 holds, for any sequence \(s_{0\rightarrow T}\), we
can always find a convex combination of all elements in it, such that
the decision maker is indifferent between the sequence and the convex
combination of its elements. By Lemma 2, I show this convex combination
is unique.

\emph{Lemma 2}. If Axiom 1-3 holds, suppose
\(s_{0\rightarrow T}\sim \sum_{t=0}^T w_t \cdot s_t\) and
\(s_{0\rightarrow T+1} \sim \sum_{t=0}^{T-1} w'_t\cdot s_t\), where
\(w_t >0\), \(w'_t>0\), \(\sum_{t=0}^Tw_t=1\),
\(\sum_{t=0}^{T+1}w'_t=1\), we must have
\(\frac{w'_0}{w_0}=\frac{w'_1}{w_1}=…=\frac{w'_T}{w_T}\).

When \(T=1\), according to Axiom 1, there exist
\(\alpha,\zeta \in (0,1)\) such that
\(s_{0 \rightarrow 1}\sim\alpha\cdot s_0 + (1-\alpha)\cdot s_1\),
\(s_{0\rightarrow 2} \sim \zeta\cdot s_{0\rightarrow 1} + (1-\zeta)\cdot s_2\).
Meanwhile, we set
\(s_{0\rightarrow 2} \sim w'_0\cdot s_0 + w'_1\cdot s_1 + (1-w'_0-w'_1)\cdot s_2\),
where \(w'_0, w'_1 > 0\).

According to Axiom 2, we must have \(1-\zeta=1-w'_0-w'_1\). So,
\(w'_1=\zeta-w’_0\).

According to Axiom 3, it can be derived that
\(s_{0\rightarrow 1} \sim \frac{w'_0}{\zeta}\cdot s_0 + (1-\frac{w'_0}{\zeta})\cdot s_1\).

Given that
\(s_{0\rightarrow 1}\sim\alpha\cdot s_0 + (1-\alpha)\cdot s_1\), suppose
\(\alpha > \frac{w'_0}{\zeta}\), we can rewrite this preference relation
as
\(s_{0\rightarrow 1}\sim(\alpha-\frac{w'_0}{\zeta})\cdot s_0 +(1-\alpha)\cdot s_1 + \frac{w'_0}{\zeta}\cdot s_0\).

If \(s_0 \succ s_1\), by applying Axiom 3, we can derive that
\((\alpha-\frac{w'_0}{\zeta})\cdot s_0 +(1-\alpha)\cdot s_1 + \frac{w'_0}{\zeta}\cdot s_0 \succ (\alpha-\frac{w'_0}{\zeta})\cdot s_1 +(1-\alpha)\cdot s_1 + \frac{w'_0}{\zeta}\cdot s_0\),
where the right-hand side, according to the above preference relation,
is indifferent from \(s_{0\rightarrow 1}\). Thus, we get a
contradiction.

Similarly, suppose \(\alpha < \frac{w'_0}{\zeta}\), we will also get a
contradiction.

Thus, \(\alpha = \frac{w'_0}{\zeta}\), which indicates
\(\frac{w'_0}{\alpha}=\frac{w'_1}{1-\alpha}=\zeta\).

We can decompose \(s_{0\rightarrow T+1}\) by

\[
\begin{aligned}
s_{0\rightarrow T+1} &\sim (1-\alpha)\cdot s_{0\rightarrow T} + \alpha\cdot s_{T+1} \\
&\sim  (1-\alpha) \zeta\cdot s_{0\rightarrow T-1} + (1-\alpha) (1-\zeta)\cdot s_{T} + \alpha\cdot s_{T+1} 
\end{aligned}
\]

Suppose there is another way to decompose \(s_{0 \rightarrow T+1}\)
using a combination of \(s_{0\rightarrow T-1}\), \(s_{T}\), and
\(s_{T+1}\). We can denote this alternative decomposition as

\[
s_{0\rightarrow T+1} \sim \beta_0\cdot s_{0\rightarrow T-1} + \beta_1\cdot s_T + \beta_2\cdot s_{T+1} 
\]

According to Axiom 2, we must have \(\alpha = \beta_2\).
=======
Thus, if Axiom 1 holds, for any sequence \(s_{0\rightarrow T}\), we can
always find a convex combination of all elements in it, such that the
decision maker is indifferent between the sequence and the convex
combination of its elements. By Lemma 2, I show this convex combination
is unique.

\emph{Lemma 2}. If Axiom 1-2 holds, then suppose
\(s_{0\rightarrow T}\sim \sum_{t=0}^T w_t \cdot s_t\) and
\(s_{0\rightarrow T+1} \sim \sum_{t=0}^{T+1} w'_t\cdot s_t\), where
\(w_t >0\), \(w'_t>0\), \(\sum_{t=0}^Tw_t=1\), and
\(\sum_{t=0}^{T+1}w'_t=1\), we must have
\(\frac{w_0}{w'_0}=\frac{w_1}{w'_1}=…=\frac{w_T}{w'_T}\).
>>>>>>> 06181f9ba6a6863cc3f71040b7af51c1e271a799

Corollary 1.

Lemma 3. If Axiom 1 and Axiom 3-4 holds, then for any
\(s_{0\rightarrow T}\) and \(s'_{0\rightarrow T}\), where
\(u(s_t)= u(s'_t)+\Delta u\) holds for any \(t\) and \(\Delta u\) is a
constant real number, we have \(w_t=w'_t\).

Suppose
\(\alpha \cdot s_t+(1-\alpha)\cdot c\sim\alpha \cdot s'_t+(1-\alpha)\cdot c'\)

<<<<<<< HEAD
From Axiom 3,
=======
From Axiom 4,
>>>>>>> 06181f9ba6a6863cc3f71040b7af51c1e271a799
\(\alpha \cdot u(s_t)+(1-\alpha)\cdot u(c)=\alpha \cdot u(s'_t)+(1-\alpha)\cdot u(c')\)

This yields \(u(s_t)-u(s'_t)=\Delta u\), where
\(\Delta u=\frac{1-\alpha}{\alpha} (u(c')-u(c))\).

By Lemma 1, if Axiom 1 holds, we have \(V(s_c)=u(c)\). The same applies
to \(V(s'_c)\).

<<<<<<< HEAD
By Axiom 4, we have
=======
By Axiom 3, we have
>>>>>>> 06181f9ba6a6863cc3f71040b7af51c1e271a799
\(V(s_{0\rightarrow T})=V(s'_{0\rightarrow T})+\Delta u\).

This yields \(\sum_{t=0}^T w_tu(s_t)- w'_tu(s'_t)=\Delta u\)

Replace \(\Delta u\), we have
\(\sum_{t=0}^T w_tu(s_t)- w'_tu(s'_t)=\sum_{t=0}^T w_t(u(s_t)- u(s'_t))\)

So, \(\sum_{t=0}^T (w_t-w'_t) u(s'_t)=0\)

<<<<<<< HEAD
Given instantaneous utility can be any non-negative real number, we must
have \(w_t=w'_t\).
=======
Given each instantaneous utility can be any non-negative real number, we
must have \(w_t=w'_t\).
>>>>>>> 06181f9ba6a6863cc3f71040b7af51c1e271a799

The FOC condition of the constrained optimal discounting problem is:

\[
<<<<<<< HEAD
f'_t(w_t)=u(x_t)+\theta,\; \forall t\in\{0,1,...,T\}
=======
f'_t(w_t)=u(s_t)+\theta,\; \forall t\in\{0,1,...,T\}
>>>>>>> 06181f9ba6a6863cc3f71040b7af51c1e271a799
\]

  \bibliography{reference.bib}

\end{document}
