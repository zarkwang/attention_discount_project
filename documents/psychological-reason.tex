% Options for packages loaded elsewhere
\PassOptionsToPackage{unicode}{hyperref}
\PassOptionsToPackage{hyphens}{url}
%
\documentclass[
  12pt,
]{article}
\usepackage{amsmath,amssymb}
\usepackage{iftex}
\ifPDFTeX
  \usepackage[T1]{fontenc}
  \usepackage[utf8]{inputenc}
  \usepackage{textcomp} % provide euro and other symbols
\else % if luatex or xetex
  \usepackage{unicode-math} % this also loads fontspec
  \defaultfontfeatures{Scale=MatchLowercase}
  \defaultfontfeatures[\rmfamily]{Ligatures=TeX,Scale=1}
\fi
\usepackage{lmodern}
\ifPDFTeX\else
  % xetex/luatex font selection
\fi
% Use upquote if available, for straight quotes in verbatim environments
\IfFileExists{upquote.sty}{\usepackage{upquote}}{}
\IfFileExists{microtype.sty}{% use microtype if available
  \usepackage[]{microtype}
  \UseMicrotypeSet[protrusion]{basicmath} % disable protrusion for tt fonts
}{}
\makeatletter
\@ifundefined{KOMAClassName}{% if non-KOMA class
  \IfFileExists{parskip.sty}{%
    \usepackage{parskip}
  }{% else
    \setlength{\parindent}{0pt}
    \setlength{\parskip}{6pt plus 2pt minus 1pt}}
}{% if KOMA class
  \KOMAoptions{parskip=half}}
\makeatother
\usepackage{xcolor}
\usepackage[margin=1in]{geometry}
\usepackage{graphicx}
\makeatletter
\def\maxwidth{\ifdim\Gin@nat@width>\linewidth\linewidth\else\Gin@nat@width\fi}
\def\maxheight{\ifdim\Gin@nat@height>\textheight\textheight\else\Gin@nat@height\fi}
\makeatother
% Scale images if necessary, so that they will not overflow the page
% margins by default, and it is still possible to overwrite the defaults
% using explicit options in \includegraphics[width, height, ...]{}
\setkeys{Gin}{width=\maxwidth,height=\maxheight,keepaspectratio}
% Set default figure placement to htbp
\makeatletter
\def\fps@figure{htbp}
\makeatother
\setlength{\emergencystretch}{3em} % prevent overfull lines
\providecommand{\tightlist}{%
  \setlength{\itemsep}{0pt}\setlength{\parskip}{0pt}}
\setcounter{secnumdepth}{5}
\usepackage{setspace} \usepackage{amsmath} \usepackage{array} \usepackage{caption} \usepackage{longtable} \usepackage{booktabs} \usepackage{enumitem} \renewcommand{\arraystretch}{1} \captionsetup[table]{skip=5pt} \setstretch{1.5}
\ifLuaTeX
  \usepackage{selnolig}  % disable illegal ligatures
\fi
\usepackage[]{natbib}
\bibliographystyle{apalike}
\IfFileExists{bookmark.sty}{\usepackage{bookmark}}{\usepackage{hyperref}}
\IfFileExists{xurl.sty}{\usepackage{xurl}}{} % add URL line breaks if available
\urlstyle{same}
\hypersetup{
  pdftitle={Psychological Reason},
  pdfauthor={Zark Zijian Wang},
  hidelinks,
  pdfcreator={LaTeX via pandoc}}

\title{Psychological Reason}
\author{Zark Zijian Wang}
\date{2023-10-02}

\begin{document}
\maketitle

\hypertarget{the-decision-process}{%
\subsection{The Decision Process}\label{the-decision-process}}

Suppose time is discrete. Let \(X_T\) denote the sequence of rewards
\([x_0,x_1,...,x_T]\), which yields reward \(x_t\) in time period
\(t\).\footnote{I use uppercase letters to represent a sequence and
  lowercase letters to represent elements within the sequence.} The time
length of this sequence, denoted by \(T\), is finite. For any
\(t \in \{0,1,...,T\}\), the reward level \(x_t\) is a random variable
defined on \(\mathbb{R}_{\geq 0}\). I assume that making an
intertemporal choice involves three steps:

\setlength{\leftskip}{1cm}

Step 1. (Sampling) The decision maker subjectively draws a few potential
realizations of \(X_T\), and from each drawn realization, she draws a
few time periods and observes their rewards; then, she combines all
observed rewards into a sample.

Step 2. (Valuation) The decision maker uses the mean utility of sampled
rewards as an approximate value representation of \(X_T\).

Step 3. (Choice-making) She chooses the sequence with the highest value
from all the available reward sequences.

\setlength{\leftskip}{0pt}

In the decision process described above, Step 3 is standard. By Step
1-2, I take the notion that, to evaluate a stimuli, the decision maker
needs to assess all the relevant information, while her information
processing capacity is limited. Consequently, she selectively attends to
only \emph{a subset} of the available information (which is termed
\emph{a sample}), then aggregates the attributes observed in the sample
to calculate the stimuli value. This sampling process is not unbiased;
on the contrary, the decision maker aims to retain more of the
information that they consider more relevant in the sample. Such a
notion has a long history in psychological research.\footnote{\citet{weber_mindful_2009}
  and \citet{chun_taxonomy_2011} provide good reviews for such studies.}
In recent years, many theories grounded in this (or similar notions)
have made significant progress in explaining choice anomalies, such as
decision field theory \citep{busemeyer_decision_1993},
decision-by-sampling \citep{stewart_decision_2006}, utility-weighted
sampling \citep{lieder_overrepresentation_2018} and efficient coding
theory \citep{heng_efficient_2020}. In the next subsection, I describe
the sampling and valuation process in detail.

  \bibliography{reference.bib}

\end{document}
