
\documentclass[12pt]{article}


\begin{document}
\begin{table}
    \caption{Robust Linear Regressions, Using Gaussian Mixture Model for Clustering}
    \vspace*{12pt}
    \centering

      \begin{tabular}{lll}
\hline
 & (1) Pooled & (2) FE \\
\hline
$X_1\cdot1\{T=T_L\}\times$CL1 & -0.026$^{***}$ & -0.01$^{***}$ \\
 & (0.003) & (0.001) \\
$X_1\cdot1\{T=T_H\}\times$CL1 & -0.029$^{***}$ & -0.013$^{***}$ \\
 & (0.003) & (0.001) \\
$X_1\cdot1\{T=T_L\}\times$CL2 & 0.021$^{***}$ & 0.002 \\
 & (0.003) & (0.002) \\
$X_1\cdot1\{T=T_H\}\times$CL2 & 0.021$^{***}$ & 0.002 \\
 & (0.003) & (0.002) \\
PELI & 0.154 & 1.73$^{***}$ \\
 & (0.776) & (0.328) \\
Constant & 54.553$^{***}$ & 52.648$^{***}$ \\
 & (0.87) & (0.37) \\\hline

observations & 2254 & 2254 \\
Muller-Welsh & 382.354 & 143.607 \\
\hline
\end{tabular}
% INSERT reg_rlm_GMM

    \vspace*{4pt}
    \centering
    \begin{minipage}{0.85\textwidth}
    {\par\footnotesize Note: * $p<0.05$, ** $p<0.01$, *** $p<0.005$. Standard errors are reported in the parentheses. Each model is estimated using Huber's M-estimator (where the threshold is set at 1.345) and the scale estimator is Huber's proposal 2 estimator. Each p-value for RLM is calculated based on a normal distribution with i.i.d. assumption. A smaller Muller-Welsh score indicates the model has a greater ability to both parsimoniously fit the data and predict new independent obeservations. $Y_1$ and $T$ denote the front-end amount and the sequence length in Option A. $T_L$ and $T_H$ are 6 months and 12 months respectively. Clustering results are obtained through Gaussian mixture model. FE denotes fixed effects.}
    \end{minipage}
    \label{tab:seq_value_rlm}
\end{table}

\end{document}

