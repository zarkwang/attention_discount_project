
\documentclass[12pt]{article}


\begin{document}
\begin{table}
    \caption{Regression Results for Count-the-Rabbits Tasks}
    \vspace*{12pt}
    \centering

      \begin{tabular}{lllll}
\hline
 & (1) Pooled & (2) FE & (3) FE & (4) FE \\
\hline
Group & -0.732$^{*}$ & -4.466$^{*}$ & -0.21 & -0.964$^{***}$ \\
 & (0.331) & (1.789) & (1.771) & (0.312) \\
Question$\cdot1\{\text{Group}=0\}$ & 0.581$^{*}$ & 1.357$^{*}$ & 1.618$^{*}$ & 2.032$^{**}$ \\
 & (0.228) & (0.606) & (0.681) & (0.778) \\
Questsion$\cdot1\{\text{Group}=1\}$ & 0.07 & 0.225 & 0.466 & 0.214 \\
 & (0.319) & (0.376) & (0.444) & (0.352) \\\hline

observations & 3504 & 3504 & 2190 & 810 \\
aic & 883.789 & 1103.793 & 752.785 & 586.582 \\
\hline
\end{tabular}
% INSERT reg_rabbit

    \vspace*{4pt}
    \centering
    \begin{minipage}{0.85\textwidth}
    {\par\footnotesize Note: * $p<0.05$, ** $p<0.01$, *** $p<0.005$. Standard errors are clustered at the subject level and are reported in the parentheses. The p-values are calculated based on Wald tests. FE denotes individual fixed-effects. Model (1)-(2) are run upon the full sample, (3) is for those having changed choices at least once in intertemporal choice tasks, (4) is for those having made wrong choices in count-the-rabbits tasks. Each model includes an intercept and task-specific dummies. }
    \end{minipage}
    \label{tab:exp3_reg_rabbit_choice}
\end{table}

\end{document}

