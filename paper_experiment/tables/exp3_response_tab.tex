
\begin{table}[!h]
    \caption{Predicting response times with choices and conditions in Experiment 1}
    \vspace*{10pt}
    \centering

      \begin{tabular}{lll}
\hline
 & Intertemporal Choice & Count-the-Rabbits \\
\hline
Group & -0.684$^{***}$ & -0.792$^{***}$ \\
 & (0.141) & (0.144) \\
Question$\cdot1\{\text{Group}=0\}$ & -0.165 & 0.912$^{***}$ \\
 & (0.174) & (0.199) \\
Question$\cdot1\{\text{Group}=1\}$ & 0.457$^{***}$ & 0.849$^{***}$ \\
 & (0.101) & (0.132) \\
Choice & 0.954$^{*}$ & 1.291$^{***}$ \\
 & (0.399) & (0.456) \\
Choice$\times$Group & -0.762$^{*}$ & -1.265$^{***}$ \\
 & (0.304) & (0.229) \\
Choice$\times$Question$\cdot1\{\text{Group}=0\}$ & 0.001 & -0.138 \\
 & (0.257) & (0.23) \\
Choice$\times$Question$\cdot1\{\text{Group}=1\}$ & -0.12 & 0.263 \\
 & (0.195) & (0.175) \\\hline

observations & 4393 & 2179 \\
AIC & 18560.034 & 8711.872 \\
adj-$R^2$ & 0.381 & 0.55 \\
\hline
\end{tabular}
% INSERT reg_response_time

    \vspace*{4pt}
    \centering
    \begin{minipage}{0.85\textwidth}
    {\par\footnotesize Note: * $p<0.05$, ** $p<0.01$, *** $p<0.005$. Both models are estimated through 2SLS method. For first-stage regression, we use the model for Column (2) in Table \ref{tab:exp3_reg_intertemporal_choice} to predict intertemporal choices, and the model for Column (3) in Table \ref{tab:exp3_reg_rabbit_choice} to predict count-the-rabbits choices. The variable Choice which is 1 if the predicted choice is the sequence option and is 0 otherwise. For second-stage regression, indepdent variables are the predictors shown in the table plus task-specific dummies and their interactions with Choice, and participant-specific dummies. Standard errors (in the parentheses) are clustered at the subject level. $p$-values are calculated based on t-tests. }
    \end{minipage}
    \label{tab:exp3_reg_response_time}
\end{table}


