% Options for packages loaded elsewhere
\PassOptionsToPackage{unicode}{hyperref}
\PassOptionsToPackage{hyphens}{url}
%
\documentclass[
  12pt,
]{article}
\usepackage{amsmath,amssymb}
\usepackage{iftex}
\ifPDFTeX
  \usepackage[T1]{fontenc}
  \usepackage[utf8]{inputenc}
  \usepackage{textcomp} % provide euro and other symbols
\else % if luatex or xetex
  \usepackage{unicode-math} % this also loads fontspec
  \defaultfontfeatures{Scale=MatchLowercase}
  \defaultfontfeatures[\rmfamily]{Ligatures=TeX,Scale=1}
\fi
\usepackage{lmodern}
\ifPDFTeX\else
  % xetex/luatex font selection
\fi
% Use upquote if available, for straight quotes in verbatim environments
\IfFileExists{upquote.sty}{\usepackage{upquote}}{}
\IfFileExists{microtype.sty}{% use microtype if available
  \usepackage[]{microtype}
  \UseMicrotypeSet[protrusion]{basicmath} % disable protrusion for tt fonts
}{}
\makeatletter
\@ifundefined{KOMAClassName}{% if non-KOMA class
  \IfFileExists{parskip.sty}{%
    \usepackage{parskip}
  }{% else
    \setlength{\parindent}{0pt}
    \setlength{\parskip}{6pt plus 2pt minus 1pt}}
}{% if KOMA class
  \KOMAoptions{parskip=half}}
\makeatother
\usepackage{xcolor}
\usepackage[margin=1in]{geometry}
\usepackage{graphicx}
\makeatletter
\def\maxwidth{\ifdim\Gin@nat@width>\linewidth\linewidth\else\Gin@nat@width\fi}
\def\maxheight{\ifdim\Gin@nat@height>\textheight\textheight\else\Gin@nat@height\fi}
\makeatother
% Scale images if necessary, so that they will not overflow the page
% margins by default, and it is still possible to overwrite the defaults
% using explicit options in \includegraphics[width, height, ...]{}
\setkeys{Gin}{width=\maxwidth,height=\maxheight,keepaspectratio}
% Set default figure placement to htbp
\makeatletter
\def\fps@figure{htbp}
\makeatother
\setlength{\emergencystretch}{3em} % prevent overfull lines
\providecommand{\tightlist}{%
  \setlength{\itemsep}{0pt}\setlength{\parskip}{0pt}}
\setcounter{secnumdepth}{5}
\usepackage{setspace} \usepackage{amsmath} \usepackage{array} \usepackage{caption} \usepackage{longtable} \usepackage{booktabs} \renewcommand{\arraystretch}{1} \captionsetup[table]{skip=5pt} \setstretch{1.5}
\ifLuaTeX
  \usepackage{selnolig}  % disable illegal ligatures
\fi
\usepackage[]{natbib}
\bibliographystyle{apalike}
\IfFileExists{bookmark.sty}{\usepackage{bookmark}}{\usepackage{hyperref}}
\IfFileExists{xurl.sty}{\usepackage{xurl}}{} % add URL line breaks if available
\urlstyle{same}
\hypersetup{
  pdftitle={Attentional Discounted Utility},
  pdfauthor={Zijian Zark Wang},
  hidelinks,
  pdfcreator={LaTeX via pandoc}}

\title{Attentional Discounted Utility}
\author{Zijian Zark Wang}
\date{June 21, 2023}

\begin{document}
\maketitle

\hypertarget{introduction}{%
\section{Introduction}\label{introduction}}

The discounted utility framework has been widely employed to model
intertemporal choices. According to this framework, decision makers
evaluate a sequence of rewards by assigning a weight to each time period
and summing up the weighted utilities of the rewards across these
periods. Typically, the weights are referred to as the discounting
factors and are assumed to decline over time, indicating a preference
for receiving a sooner reward over a later reward. However, recent
research suggests that, due to limited information processing capacity,
prior to making a decision, people tend to allocate more attention to
the information most relevant to the decision. We propose that the
determination of weights in discounted utility framework is also
influenced by such an attentional mechanism. For instance, when an
investor consider whether to invest \$100 now and get \$110 in 100 days,
she may focus more on the amount of money she can obtain on the 100th
day (i.e.~£110). Consequently, when calculating the utility of
``receiving £110 in 100 days'', the investor may want to assign a higher
weight to the 100th day. And doing that will reduce weight assigned to
the days prior to the 100th day. Based on this perspective, we develop a
novel model of intertemporal choice, which we term ``\emph{attentional
discounted utility}''. We demonstrate how this model can better
accommodate certain empirical findings. To distinguish it from the
discounting factor, we refer to the weights in our model as ``attention
weights.''

Let \(\textbf{x} = [x_0, x_1, …, x_T]\) denote a sequence of rewards and
\(u(.)\) denote the utility function. The overall utility of this
sequence is calculated by \(\sum_{t=0}^T w_t u(x_t)\), where \(w_t\) is
the attention weight assigned to period \(t\). In our model, the
attention weight is calculated by

\[
w_t = \frac{d_t e^{u(x_t)}}{\sum_{\tau=0}^T d_\tau e^{u(x_t)}}
\]

where \(d_t\) is the initial weight (discounting factor) allocated to
period \(t\). The weight for each period, i.e.~\(w_t\), is increasing
with \(u(x_t)\), indicating that the decision maker is motivated to
shift attention to the periods with larger rewards. \(w_t\) is
``anchored'' in the initial weight \(d_t\), because the attention
adjustment process is costly. The sum of weights is fixed at 1,
indicating the decision maker's capacity of focusing is limited.

We draw inspirations on three fields of research. The first is motivated
beliefs. The theory of motivated beliefs states that people will adjust
beliefs to subjectively maximize their utility. When the choice made by
a decision maker does not deliver the maximum utility, she may tend to
adjust the belief to safeguard the choice, rather than shift to another
option. This is usually applied to cognitive dissonance and
overconfidence. The similar mechanism may exist in the allocation of
weights to each time period. \textbf{ostrich effect}

The second is rational inattention. The model of rational inattention
states that, prior to making choices between options, the decision maker
has to learn the information about each option. Under the assumption
that she wants to maximum the expected utility minus the cost of
learning, which is linear to the information gains, we can derive that
the probability of each option being chosen follows a logistic-like
distribution. In our model, we assume that when allocating weights
across time periods, the decision maker has a similar objective
function, thus the attention weights also follows a logistic-like
distribution.

The third is the related evidence in intertemporal choices. (1) hidden
zero effect (2) concentration bias (3) models of magnitude-increasing
patience.

In the attention-adjusted discounted utility model (hereafter referred
to as ``ADU''), the underlying cognitive process is efficient sampling.
We assume that the decision maker initially has no information about
which period in a reward sequence has a larger reward. She implements a
costly sampling strategy to draw some rewards from the sequence to learn
the information, then choose the attention weight to each period
accordingly. Therefore, when evaluating the given reward sequence, she
tends to assign more weights (pay more attention) to the time periods
with larger rewards, in order to subjectively maximize her overall
utility. This attention adjustment process incurs a cognitive cost; and
the more the weight allocation deviates from the initial allocation, the
greater the cost is. The decision maker optimally re-allocates the
weights across time periods.

Our contribution is three-fold. First, the model can accommodate a lot
of empirical evidence with just one to two parameters.

In this paper, I show that a set of intertemporal choice anomalies can
be attributed to such attention adjustment processes (that is, can be
explained by ADU to some extent), including common difference effect and
magnitude effect \citep{loewenstein_anomalies_1992}, risk aversion over
time lotteries \citep{onay_intertemporal_2007, dejarnette_time_2020},
non-additive time intervals
\citep{read_is_2001, scholten_discounting_2006}, intertemporal
correlation aversion \citep{andersen_multiattribute_2018}, and dynamic
inconsistency. The model can also offer insights on the preferences for
sequences of outcomes \citep{loewenstein_preferences_1993} and the
formation of reference-dependent preferences \citep{koszegi_model_2006}.
In an empirical test, I find ADU outperforms a set of time discounting
models in predicting human intertemporal choices. Therefore, I think
there is a need to rethink the foundation of many behavioral phenomena.

Second, we link the literature of attention to time preferences.

Third, we make some novel predictions based on the model.

The remaining part of this document is organized as follows. Section
\ref{model} outlines the model of attention-adjusted discounted utility
(ADU). Section \ref{behavioral} explains how the model can help explain
some empirical findings in intertemporal choice.

\hypertarget{the-model}{%
\section{\texorpdfstring{The Model
\label{model}}{The Model }}\label{the-model}}

\hypertarget{rational-inattention-and-time-discounting}{%
\subsection{Rational Inattention and Time
Discounting}\label{rational-inattention-and-time-discounting}}

Consider a reward sequence \(x = [x_0,x_1,...,x_T]\) that yields reward
\(x_t\) in time period \(t\). The time length of this sequence, denoted
by \(T\), is finite. For any \(t \in \{0,1,...,T\}\), the reward level
\(x_t\) is a random variable defined on \(R_{+}\). The support of \(x\)
is \(X\), which is a subset of \(R_{+}^T\).

Suppose a decision maker evaluates reward sequence \(x\) by three steps:
At first, she randomly draws some potential realizations of \(x\) from
\(X\). Then, from each drawn realization of \(x\), she draws some time
periods at random, taking the rewards of these periods into a sample.
Finally, she uses the mean utility of sampled rewards as a value
representation of \(x\). Let \(s=[s_0,s_1,...,s_T]\) be a potentially
realized outcome of \(x\) and \(p(s)\) be the probability that \(s\) is
drawn. I use \(w(.)\) and \(u(.)\) to denote the decision maker's weight
function and utility function, where \(w(s_t)\) is the probability that
the reward of the \(t\)-th period in a potentially realized sequence
\(s\) is sampled, \(u(s_t)\) is the utility obtained by reward \(s_t\)
(\(t \in \{0,1,...,T\}\)), \(u'>0\), \(u''<0\).

The sampling process is sequential, and the decision maker wants to find
a sampling strategy, denoted by function \(w(.)\), that maximizes her
overall utility. In a given potentially realized sequence \(s\), the
periods with larger reward levels should be sampled more frequently.
However, at the very beginning, the decision maker has no information
about which period in \(s\) has a larger reward -- she learns such
information gradually in the process of sampling. This learning process
triggers a cognitive cost. Hence, her overall utility is the mean
utility of sampled rewards minus the cognitive cost of learning.

Suppose when having no information, the weight on period \(t\) across
each potentially realized sequence is equal (\(\equiv w^0_t\)). Let
\(W\) and \(P\) be the minimal sets that contain all available function
\(w\) and \(p\) respectively. We can use an optimization problem to
represent the described evaluation procedure: \[ 
\begin{aligned}
\max_{w\in W}  \quad & \sum_{s\in X}\sum_{t=0}^T w(s_t)u(s_t) - C(w;\theta) \\
s.t. \quad &  \sum_{s\in X}\sum_{t=0}^T w(s_t)=1 \\
& w(s_t)>0, \forall s\in X,t=0,1,…,T \\
\end{aligned}
\]where \(C(.)\) is a cognitive cost function with \(\theta\) as its
parameters. To solve this optimization problem, I add two additional
assumptions. The first is that the weight updating process is consistent
with Bayes rule, that is, \(w^0_t=\sum_{s\in X} w(s_t)\). The second is
that the cognitive cost function takes a form similar to Shannon mutual
information, that is\[
C(\textbf{w};\theta)= \lambda \sum_{s\in X}\sum_{t=0}^T w(s_t) \log\left(\frac{w(s_t)}{p(s)w_t^0}\right)
\]where \(p(s)w^0_t\) is the probability of \(s_t\) being sampled when
no information is learned, \(w(s_t)\) is the probability of that after
learning the information about \(x\). Shannon mutual information
quantifies the amount of information gain when learning about which time
period has a larger reward in any initially unknown \(s\). Consistent
with \citet{matejka_rational_2015}, I set \(C(\textbf{w};\theta)\)
linear to that. Parameter \(\lambda\) denotes unit cost of information
(\(\lambda>0\)).

Define \(w(s_t|s) = \frac{w(s_t)}{p(s)}\). As is shown in
\citet{matejka_rational_2015}, the optimization problem can be easily
solved by Lagrangain method. The solution is\[ \tag{1}
w(s_t|s) =\frac{w_t^0e^{u(s_t)/\lambda}}{\sum_{t=0}^T w_\tau^0 e^{u(s_t)/\lambda}}
\]Note \(w(s_t|s)\) reveals how the decision maker weights the utility
of time period \(t\) in a drawn sequence \(s\). It can naturally
represent the discounting factor. \(w(s_t|s)\) is increasing in \(s_t\),
which implies the decision maker exhibit more patience for a larger
reward.

While building the model, I was mainly inspired by the theories of
rational inattention
\citep{matejka_rational_2015, jung_discrete_2019, mackowiak_rational_2023}.
In \citet{matejka_rational_2015}'s theory of rational inattention, the
decision maker makes choices between discrete alternatives; she
evaluates each alternative via a costly information acquisition process,
then decides the optimal choice strategy. The theory deduces the
probability of each alternative being chosen should follow a
logistic-like distribution. In ADU, I assume the discounting factors are
generated by a similar process; hence, she subjectively weights each
time period according to a logistic-like distribution -- as Equation (1)
does -- as well.

The reason why I use Shannon mutual information as the cognitive cost
function is twofold. First, note that
\(w(s_t|s) \propto w^0_t e^{u(s_t)/\lambda}\). Given a certain stream
\(s\) and two time periods \(t_1\) and \(t_2\) (\(t_2>t_1\)), the
relative weight between them \(\frac{w(s_{t_1}|s)}{w(s_{t_2}|s)}\) is
only relevant to \(s_{t_1}\) and \(s_{t_2}\). Therefore, changing the
reward of a third period has no impact on how the reward in \(t_2\)
should be discounted relative to that in \(t_1\). Second, under such
settings, the objective function can be rewritten as\[
\sum_{s\in X} p(s)[w(s_t|s)u(s_t) - \lambda D_{KL}]
\]

where \(D_{KL}\) is the KL divergence between the initial weights over
time periods and the weights updated given the stream \(s\) is drawn.
Clearly, the determination of \(w(s_t|s)\) in each \(s\) can be
separated from each other. In other words, given two potentially
realized streams \(s\) and \(s'\), the changes in \(s'\) has no impact
on the determination of discounting factors in \(s\). This property is
consistent with many forms of optimal sequential learning (for example,
\citet{zhong_optimal_2022} ). \citet{matejka_rational_2015} and
\citet{caplin_rationally_2022} show that the two properties are jointly
satisfied if and only if the solution of \(w(s_t|s)\) follows Equation
(1).

\hypertarget{the-rationale-behind-mutual-information}{%
\subsection{The Rationale Behind Mutual
Information}\label{the-rationale-behind-mutual-information}}

Axiom 1: (state independence) the probability weighting process and time
discounting process can be separated each other.

Axiom 2: (spread-consistency correlation) when allocating a consumption
budget across time periods, consumers keep their choices dynamically
consistent if and only if they perform a strong preference for spread.

Proposition 1: If attention weights are generated by the process, the
weights take the logistic form if and only if the cognitive cost
function is linear to Shannon mutual information.

\hypertarget{implications-in-time-preferences}{%
\section{\texorpdfstring{Implications in Time Preferences
\label{behavioral}}{Implications in Time Preferences }}\label{implications-in-time-preferences}}

\hypertarget{valuation-of-a-delayed-reward}{%
\subsection{Valuation of A Delayed
Reward}\label{valuation-of-a-delayed-reward}}

Suppose a decision maker receives a positive detereminstic reward in
time period \(j\) (and no reward in other periods), that is, for any
\(t\in[0,1,…,T]\) and \(t \neq j\), \(x_t = 0\). The decision maker
evaluates the reward sequence by implementing the ADU evaluation
procedure. For simplicity, I set \(v(0)=0\). I also set the decision
maker initially holds stationary time preferences,
i.e.~\(w^0_t=\delta^t\), where \(\delta\in(0,1]\). When \(\delta=1\), we
say that the initial attention is uniformly distributed across time
periods. Given the reward is detereminsic, one can omit \(s\) in
\(w(s_t|s)\) and directly represent the weight on each time period \(t\)
by \(w_t\).

\[ 
w_j = \left\{ \begin{aligned}
& \delta^j \cdot\frac{1}{1+\frac{\delta}{1-\delta}(1-\delta^T)e^{-v(x_j)}}\;, & 0<\delta<1 \\
& \frac{1}{1+T\cdot e^{-v(x_j)}}\; , & \delta=1
\end{aligned}
\right.
\]

\hypertarget{hidden-zero-effect}{%
\subsection{Hidden Zero Effect}\label{hidden-zero-effect}}

One way to validate the model is that, when we frame of the length of
sequence in different ways, the decision maker's overall utility may
change.

One evidence is \citet{magen_hidden-zero_2008}, which find that people
perform greater patience when both SS and LL are framed as sequences,
rather than being framed single-period rewards. They term this finding
as ``hidden zero effect''. For instance, suppose SS is ``receive £100
today'' and LL is ``receive £120 in 6 months'', and we have

SS\textsubscript{0}: ``receive £100 today and £0 in 6 months''

LL\textsubscript{0}: ``receive £0 today and £120 in 6 months''

Then people will be more likely to prefer LL\textsubscript{0} over
SS\textsubscript{0} than preferring LL over SS. The subsequent studies
(e.g. \citet{read_value_2017}) show that the hidden zero effect is
asymmetric. That is, shifting SS to SS\textsubscript{0} and keeping LL
unchanged leads to an increase in patience, whereas shifting LL to
LL\textsubscript{0} and keeping SS unchanged cannot increase patience.
The attention-based explanation is that, in SS, the decision maker may
conceive the length of sequence as ``today''; in SS\textsubscript{0},
she may conceive the length as ``6 months''. In the latter case, she
allocates some attention weights to some future periods with zero
reward, which decreases her overall utility.

The existence of hidden zero effect also provides some hints on the
selection of time length \(T\). As is shown in Equation
\ref{eq:w_delay}, when evaluating a delayed reward delivered in period
\(j\), the range of \(T\) is \([j,+\infty)\). An increase in \(T\) will
reduce the overall utility. Thus, when comparing SS and LL, the decision
maker may tend to set \(T=j\) (the minimum length she can set), in order
to optimize the overall utility. That is, without mentioning the periods
after \(j\), she does not necessarily sample from the periods later than
\(j\) to evaluate the given delayed reward. In this case, i.e.~\(T=j\)
we have

\[
w_T = \frac{1}{1+G(T)e^{-v(x_T)}}
\]

where

\[
G(T) = \left\{ \begin{aligned}
& \frac{1}{1-\delta}(\delta^{-T}-1) \; ,& 0<\delta<1\\
& T\; ,& \delta=1\
\end{aligned}
\right.
\]

Clearly, when \(\delta=1\), the attention weight \(w_j\) takes a form
similar with hyperbolic discounting.

\hypertarget{common-difference-effect}{%
\subsection{Common Difference Effect}\label{common-difference-effect}}

Suppose there are a large later reward \(x_l\) arriving at period
\(t_l\) (denoted by LL) and a small sooner reward \(x_s\) arriving at
period \(t_s\) (denoted by SS), where \(x_l>x_s>0\), \(t_l>t_s>0\).
Assuming \(w_{t_l}(x_l)v(x_l)=w_{t_s}(x_s)v(x_s)\), common difference
effect implies
\(w_{t_l+\Delta t}(x_l)v(x_l)>w_{t_s+\Delta t}(x_s)v(x_s)\) for any
positive integer \(\Delta t\)\citep{loewenstein_anomalies_1992}.

\textbf{Proposition 2}: If the initial weights are uniformly
distributed, then the common difference effect always holds; if the
initial weights exponentially declines over time, the common difference
effect holds when
\(v(x_l)-v(x_s)+\ln\frac{v(x_l)}{v(x_s)}>-(t_l-t_s)\ln\delta\).

When \(\delta = 1\), ADU predicts that the decision maker always
performs common difference effect. This is obvious because the
discounting factor \(w_T\) takes a hyperbolic-like form. When
\(\delta<1\), the decision maker performs common difference effect only
when the difference between \(x_l\) and \(x_s\) are much larger than the
difference between \(t_l\) and \(t_s\).

The ADU's prediction on common difference effect can be understood as
follows. Note that \(w_t \propto \delta^t e^{u(x_t)/\lambda}\). If we
omit the constraint that the sum of weights on each time period is fixed
(i.e.~attention is limited), then
\(w_{t_l+\Delta t}(x_l) = \delta^{\Delta t} \cdot w_{t_l}\) and the same
can be applied to \(w_{t_s+\Delta t}\). Thus,
\(w_{t_l+\Delta t} / w_{t_s+\Delta t}\) keeps constant for any
\(\Delta t\). However, given the decision maker's attention is limited,
the change from \(w_{t_l}\) to \(w_{t_l+\Delta t}\) is not only driven
by \(\delta^{\Delta t}\), but also driven by the effect that the final
period, with a positive reward, can naturally grab attention from the
previous periods which has no reward. Since \(x_l > x_s\), this
attention-grabbing effect is greater for LL than for SS. Meanwhile, when
extending the time length, the average attention that can be allocated
to each period should shrink. The decision makers performs common
difference effect only when the former effect exceeds the latter effect.

\hypertarget{magnitude-effect}{%
\subsection{Magnitude Effect}\label{magnitude-effect}}

Assuming we have \(t_l\), \(t_s\), \(x_s\) fixed, and want to find a
\(x_l\) such that \(w_{t_l}(x_l)v(x_l) = w_{t_s}(x_s)v(x_s)\), the
magnitude effect implies that, if we increase \(x_s\), then the
\(x_l/x_s\) that makes the equality valid will decrease.

\textbf{Proposition 3}: The magnitude effect always holds when the
utility function \(v(x)\) satisfies\[
RRA_v(x)\leq 1-\frac{e_v(x)}{v(x)+1}
\]where \(RRA_v(x)\) is the relative risk aversion coefficient of
\(v(x)\), \(e_v(x)\) is the elasticity of \(v(x)\) to \(x\).

\textbf{Corollary 1}: Suppose \(v(x)=x^\gamma/\lambda\), where
\(\gamma>0\) and \(\lambda>0\). Then magnitude effect always holds.

\hypertarget{concavity-of-time-discounting}{%
\subsection{Concavity of Time
Discounting}\label{concavity-of-time-discounting}}

\textbf{Proposition 4}: If \(\delta =1\), then the discount function is
convex in \(t\). If \(0<\delta<1\), then there are a reward threshold
\(\underline{x}\) and a time threshold \(\underline{t}\) such that

\begin{enumerate}
\def\labelenumi{\arabic{enumi}.}
\tightlist
\item
  when \(x\leq \underline{x}\), the discount function is convex in
  \(t\);
\item
  when \(x > \underline{x}\), the discount function is convex in \(t\)
  given \(t\geq \underline{t}\), and it is concave in \(t\) given
  \(t<\underline{t}\).
\end{enumerate}

It can be derived that \(v(\underline{x})=\ln(\frac{2}{1-\delta})\), and
\(\underline{t}=\frac{\ln[(1-\delta)e^{v(x)}-1]}{-\ln\delta}\).

Both exponential and hyperbolic discounting models predict the decision
maker is risk seeking over time lotteries. That is, suppose a
deterministic reward of level \(c\) (\(c>0\)) is delivered in period
\(t_s\) with probability \(\pi\) and is delivered in period \(t_l\) with
probability \(1-\pi\) (\(0<\pi<1\)); another deterministic reward, of
the same level, is delivered in a certain period
\(\pi t_s +(1-\pi) t_l\). The decision maker should prefer the former
case to the latter case. However, \citet{onay_intertemporal_2007} find
in experiments that people are only risk seeking over time lotteries
when \(\pi\) is small and are risk averse over time lotteries when
\(\pi\) is large. This finding can be explained by the convexity of
\(w_T\).

Let \(t_m = \pi t_s +(1-\pi) t_l\). By definition, the decision makers
are risk seeking over time lotteries when
\(\pi w_{t_s}(c)+(1-\pi)w_{t_l}(c)>w_{t_m}(c)\). First, note the LHS
equals to the RHS when \(\pi=0\) or \(\pi=1\). Fixing \(t_s\) and
\(t_l\), the inequality implies \(w_{t_m}(c)\) is convex in \(t_m\).
Second, it can be proved that \(w_T(c)\) is convex in \(T\) if and only
if \(T\) is above a certain threshold. This is also consistent with
\citet{takeuchi_non-parametric_2011} that suggests the discount function
should be inverse S-shaped with respect to time. By contrast, in many
models such as exponential and hyperbolic discounting, discounting
factors are typically decided by a convex function of \(T\). Third, note
\(t_m\) is linearly decreasing with \(\pi\), thus the decision maker is
more likely to be risk seeking over time lotteries when \(\pi\) is
small. The same can be applied to the risk aversion case.

Now consider \(T\) is small enough to make \(w_T\) concave in \(T\). In
this case, adding an extension to \(T\) will increase the rate at which
\(w_T\) declines with \(T\) -- this property is termed ``super-additive
time intervals'' by \citet{read_is_2001}. Moreover, ADU predicts
intervals are sub-additive when the total time length \(T\) is large,
and are super-additive when \(T\) is small, which is consistent with
\citet{scholten_discounting_2006}.

\hypertarget{violation-of-diminishing-sensitivity}{%
\subsection{Violation of Diminishing
Sensitivity}\label{violation-of-diminishing-sensitivity}}

We discuss two behavioral implications of this property. The first is
reference-dependent preferences. The second is sub-additivity and
super-additivity of time intervals.

\textbf{Proposition 5}: Suppose \(-\frac{v''(x)}{v'(x)^2}<2\) at
\(x=0\), then there exists a threshold \(x^*\) in \((0,+\infty)\) such
that \(V(x,t)\) is convex in \(x\) when \(x\leq x^*\) and concave in
\(x\) when \(x>x^*\).

Supporting evidence: sub- and super-additive intervals

ADU predicts that the larger the unit cost of information \(\lambda\) or
the smaller the magnitude of \(x_l\) and \(x_s\) is, the more likely it
is that the decision maker performs magnitude effect.

First, note that the magnitude effect requires the decision maker's
overall utility \(w_T(x_T)u(x_T)\) to be a convex function of \(x_T\).
Given that \(u(.)\) is concave, whether the magnitude effect holds
should depend on \(w_T\). Then, set \(z = u(x_T)-\lambda\log G(T)\). We
can rewrite Equation (2) as a logistic function of \(z\), i.e.
\(w_T = 1/(1+e^{-z/\lambda})\). By the shape of logistic function,
\(w_T\) is convex in \(u(x_T)\) if and only if
\(u(x_T)<\lambda \log G(T)\) (that is, when \(x_T\) is small relative to
\(T\) or when \(\lambda\) is large). Finally, it is notable that the
given condition is necessary but not sufficient to yield magnitude
effect.

In summary, holding the others equal, the decision makers' overall
utility can be convex in a future reward when the level of it is under a
certain threshold, and be concave when it is above the threshold. This
is also consistent to the theories about reference-dependent preferences
\citep{koszegi_model_2006}.

\hypertarget{inseparable-sequences}{%
\subsection{Inseparable Sequences}\label{inseparable-sequences}}

Let \(x\) and \(y\) denote two 2-period risky reward sequences. For
\(x\), the realized sequence is {[}£100,£100{]} with probability 1/2,
and is {[}£3,£3{]} with probability 1/2. For \(y\), the realized
sequence is {[}£3,£100{]} with probability 1/2, and is {[}£100,£3{]}
with probability 1/2. Classical models of intertemporal choice, such as
\citet{fishburn_time_1982}, typically assume the separability of
potentially realized sequences. This implies that the decision maker is
indifferent between \(x\) and \(y\). However,
\citet{andersen_multiattribute_2018} find evidence of intertemporal
correlation aversion, that is, people often prefer \(y\) to \(x\). Such
a property is also termed ``weak separability'' in
\citet{noor_constrained_2023}.

ADU can naturally yield intertemporal correlation aversion. For
simplicity, suppose the initial attention is uniformly distributed
across the two periods. For \(x\), under each potentially realized
sequence, the decision maker equally weights each period. For \(y\),
decision maker tends to assign more weight to the period with a reward
of £100 (suppose that weight is \(w\)). Then the value of \(x\) is
\(\frac{1}{2} u(100) + \frac{1}{2} u(3)\) and the value of \(y\) is
\(w\cdot u(100) +(1-w) \cdot u(3)\). Given that \(x>\frac{1}{2}\), the
decision makers should strictly prefer \(y\) to \(x\).

\hypertarget{implications-in-dynamically-inconsistent-planning}{%
\section{Implications in Dynamically Inconsistent
Planning}\label{implications-in-dynamically-inconsistent-planning}}

\hypertarget{weight-updating-and-attention-grabbing-effect}{%
\subsection{Weight Updating and Attention-Grabbing
Effect}\label{weight-updating-and-attention-grabbing-effect}}

Suppose a decision maker has budget \(m\) (\(m>0\)) and is considering
how to spend it over different time periods. We can use a reward
sequence \(x\) to represent this decision problem, where the decision
maker's spending in period \(t\) is \(x_t\). In period 0, she wants to
find a \(x\) such that\[ \tag{3}
\max_{x}\;\sum_{t=0}^T w_t u(x_t)\quad s.t. \;\sum_{t=0}^T x_t = m  
\]

where \(w_t\) is the attention-adjusted discounting factor in period
\(t\). I assume
\(w_t=\delta^t e^{u(x_t)/\lambda}/\sum_{t=\tau}^T \delta^{\tau} e^{u(x_\tau)/\lambda}\)
and there is no risk under this setting.

In models like exponential and hyperbolic discounting, the discounting
factor of a future period is consistently smaller than that of the
current period. Thus, the decision maker should spend more at the
present than in the future. By contrast, in ADU, when increasing the
spending in a certain period, the discounting factor corresponding to
that period should also increase. So it is possible that the decision
maker spends more in the future and that a future period has a greater
discounting factor than the current period. This is consistent with
\citet{loewenstein_preferences_1993} that find people sometimes prefer
improving sequences to declining sequences.

ADU suggests there are two mechanisms that can help explain why people
may perform dynamically inconsistent behavior. The first is
\emph{attention-grabbing effect}, that is, keeping the others equal,
when we increase \(x_t\) (which lead to an increase in \(w_t\)), the
discounting factor in any other period should decrease due to limited
attention. After omitting a previous period from the decision problem in
Equation (3), the decision maker can assign more weights to remaining
periods; thus, the attention-grabbing effect is enhanced. The increased
attention-grabbing effect will offset some benefit of increasing
spending toward a certain period. Therefore, when the decision maker
prefers improving sequences, the attention-grabbing effect will make her
perform a present bias-like behavior (always feeling that she should
spend more at the present than the original plan); when the decision
maker prefers declining sequences, this effect will maker her perform a
future bias-like behavior (always feeling she should spend more in the
future).

The second mechanism is \emph{initial attention updating}. As is assumed
above, in period 0, prior to evaluating each reward sequence, the
decision maker's initial weight on period \(t\) is proportional to
\(\delta^t\); after evaluation, the weight becomes being proportional to
\(\delta^t e^{u(x_t)/\lambda}\). In period 1, if she implements the
evaluation based on the information attained in period 0, the initial
weight should be updated to being proportional
\(\delta^t e^{u(x_t)/\lambda}\); thus, the weight after evaluation
should become being proportional to \(\delta e^{2u(x_t)/\lambda}\). As a
result, the benefit of increasing spending toward a certain period gets
strengthened. The updated initial attention can make those who prefer
improving sequences perform present bias and those who prefer declining
sequences perform future bias.

Both the attention-grabbing effect and initial attention updating are
affected by the curvature of utility function. They jointly decide which
behavior pattern that people should perform in dynamics.

\hypertarget{excess-smoothness-and-sensitivity-of-consumption}{%
\subsection{Excess Smoothness and Sensitivity of
Consumption}\label{excess-smoothness-and-sensitivity-of-consumption}}

\hypertarget{empirical-analysis}{%
\section{Empirical Analysis}\label{empirical-analysis}}

\hypertarget{discussion}{%
\section{Discussion}\label{discussion}}

\hypertarget{comparison-with-other-related-models}{%
\subsection{Comparison With Other Related
Models}\label{comparison-with-other-related-models}}

The third is the attention mechanisms which have been widely applied in
deep learning. In such models, there is often an input sequence and a
query vector. Attention weights are assigned to each period of the
sequence to determine their relevance to the current context. The most
common approach to obtain attention weights involves computing
similarity scores between the query vector and each period in the input
sequence, normalizing these scores using a softmax (i.e.~multinominal
logistic) function.

In addition to ADU, there are other models that attempt to incorporate
attention mechanism into the formation of time preferences. For example,
\citet{steiner_rational_2017} consider a decision maker adjusting the
belief \(p(s)\) over time but holding the discounting factor
\(w(s_t|s)\) constant. In each time period, given that her ability to
learn new information is limited, the updated belief cannot deviate from
that in a previous period by too much, which causes behavioral inertia.
Instead, ADU assumes the decision maker re-allocates \(w(s_t|s)\) each
time period. Thus, the process of attention adjustment not only affects
dynamic decision-making but also affects the choices in ``Money Earlier
or Later'' (MEL) tasks. Besides, \citet{gabaix_myopia_2017} assume the
perception of future rewards is noisy and the decision maker infers the
value of them by sampling from normal distributions;
\citet{gershman_rationally_2020} allow the decision maker optimally
chooses sample variance to minimize the mean sample squared error. Such
theories, together with a certain specification on rate-distortion
function, can lead to magnitude-increasing patience and hyperbolic-like
discounting. Discounting factors in this style can be viewed as a
special case of those in ADU. \citet{noor_optimal_2022},
\citet{noor_constrained_2023} construct an optimization problem similar
to ADU. However, they use a different cognitive cost function. I compare
the performance of ADU with models of \citet{gershman_rationally_2020},
\citet{noor_optimal_2022} and some other papers in predicting human
choices in MEL tasks.

\hypertarget{the-sampling-process-underlying-adu}{%
\subsection{The Sampling Process Underlying
ADU}\label{the-sampling-process-underlying-adu}}

\hypertarget{potential-research-directions}{%
\subsection{Potential Research
Directions}\label{potential-research-directions}}

\begin{itemize}
\item
  direct measure of discounting
\item
  intransitive time preference
\item
  range-dependent weighting (focusing) / consider salience and
  similarity
\end{itemize}

\renewcommand\refname{Reference}
  \bibliography{reference.bib}

\end{document}
