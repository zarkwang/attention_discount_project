% Options for packages loaded elsewhere
\PassOptionsToPackage{unicode}{hyperref}
\PassOptionsToPackage{hyphens}{url}
%
\documentclass[
  12pt,
]{article}
\usepackage{amsmath,amssymb}
\usepackage{iftex}
\ifPDFTeX
  \usepackage[T1]{fontenc}
  \usepackage[utf8]{inputenc}
  \usepackage{textcomp} % provide euro and other symbols
\else % if luatex or xetex
  \usepackage{unicode-math} % this also loads fontspec
  \defaultfontfeatures{Scale=MatchLowercase}
  \defaultfontfeatures[\rmfamily]{Ligatures=TeX,Scale=1}
\fi
\usepackage{lmodern}
\ifPDFTeX\else
  % xetex/luatex font selection
\fi
% Use upquote if available, for straight quotes in verbatim environments
\IfFileExists{upquote.sty}{\usepackage{upquote}}{}
\IfFileExists{microtype.sty}{% use microtype if available
  \usepackage[]{microtype}
  \UseMicrotypeSet[protrusion]{basicmath} % disable protrusion for tt fonts
}{}
\makeatletter
\@ifundefined{KOMAClassName}{% if non-KOMA class
  \IfFileExists{parskip.sty}{%
    \usepackage{parskip}
  }{% else
    \setlength{\parindent}{0pt}
    \setlength{\parskip}{6pt plus 2pt minus 1pt}}
}{% if KOMA class
  \KOMAoptions{parskip=half}}
\makeatother
\usepackage{xcolor}
\usepackage[margin=1in]{geometry}
\usepackage{graphicx}
\makeatletter
\def\maxwidth{\ifdim\Gin@nat@width>\linewidth\linewidth\else\Gin@nat@width\fi}
\def\maxheight{\ifdim\Gin@nat@height>\textheight\textheight\else\Gin@nat@height\fi}
\makeatother
% Scale images if necessary, so that they will not overflow the page
% margins by default, and it is still possible to overwrite the defaults
% using explicit options in \includegraphics[width, height, ...]{}
\setkeys{Gin}{width=\maxwidth,height=\maxheight,keepaspectratio}
% Set default figure placement to htbp
\makeatletter
\def\fps@figure{htbp}
\makeatother
\setlength{\emergencystretch}{3em} % prevent overfull lines
\providecommand{\tightlist}{%
  \setlength{\itemsep}{0pt}\setlength{\parskip}{0pt}}
\setcounter{secnumdepth}{5}
\usepackage{setspace} \usepackage{amsmath} \usepackage{array} \usepackage{caption} \usepackage{longtable} \usepackage{booktabs} \usepackage{enumitem} \renewcommand{\arraystretch}{1} \captionsetup[table]{skip=5pt} \setstretch{1.5}
\ifLuaTeX
  \usepackage{selnolig}  % disable illegal ligatures
\fi
\usepackage[]{natbib}
\bibliographystyle{apalike}
\IfFileExists{bookmark.sty}{\usepackage{bookmark}}{\usepackage{hyperref}}
\IfFileExists{xurl.sty}{\usepackage{xurl}}{} % add URL line breaks if available
\urlstyle{same}
\hypersetup{
  pdftitle={Progress Report for Annual Review 22/23 - v2},
  pdfauthor={Zark Zijian Wang},
  hidelinks,
  pdfcreator={LaTeX via pandoc}}

\title{Progress Report for Annual Review 22/23 - v2}
\author{Zark Zijian Wang}
\date{August 05, 2023}

\begin{document}
\maketitle

\hypertarget{introduction}{%
\section{Introduction}\label{introduction}}

In this document, I introduce three pieces of work.

The first is a model I term ``attentional discounted utility''. I
leverage attentional mechanism to explain the anomalies in intertemporal
choice. The model is developed based on expected discounted utility
framework.

The second is an experimental test on whether (and how) attentional
mechanism can influence intertemporal choice.

The third is an experiment aiming to test how people valuate risky
outcomes and efforts. One prominent application of the findings from
this experiment is that to help us validate a key hypothesis in rational
inattention theories. I have designed the experimental program.

\hypertarget{attentional-discounted-utility}{%
\section{Attentional Discounted
Utility}\label{attentional-discounted-utility}}

\hypertarget{the-model}{%
\subsection{The Model}\label{the-model}}

Expected discounted utility framework has been widely used in behavioral
and economic research. Given a sequence of rewards
\(X_T=[x_0,x_1,...,x_T]\), which yields reward \(x_t\) in time period
\(t\),\footnote{I use uppercase letters to represent a sequence and
  lowercase letters to represent each element within the sequence.}
\(t \in \{0,1,...,T\}\), the expected discounted utility of \(X_T\) is
then calculated by\[
EDU(X_T)= \mathbb{E}\left[\sum_{t=0}^T d_t u(x_t)\right]
\]where \(d_t\) is the discounting factor of time period \(t\), reward
level \(x_t\) is a random variable defined on \(\mathbb{R}_{\geq 0}\),
\(u(.)\) denote the decision maker's instantaneous utility function,
where \(u'>0\) and \(u''<0\). The time length of this sequence, denoted
by \(T\), is finite.

I define a class of models that incorporates attentional mechanism into
such a valuation process of sequential rewards. In such models, a
decision maker allocates attention across sequential outcomes when
processing information about rewards, and attends more to the outcomes
with larger rewards. I refer to this class of models as
\emph{Attentional Discounted Utility} (ADU). Previous studies exploring
the ADU setting include the works by \citet{gershman_rationally_2020}
and \citet{noor_optimal_2022}. Specifically, I focus on a particular
variant of ADU, which I term as \emph{ADU with Shannon cost function}
(ADUS). ADUS retains the architecture of expected discounted utility,
but modifies the conventional discounting factor \(d_t\) to \(w_t\),
which I refer to as attention weight hereafter, where

\begin{equation}\tag{1}\label{eq:wt}
w_t = \frac{d_t e^{u(x_t)/\lambda}}{\sum_{\tau=0}^T d_\tau e^{u(x_\tau)/\lambda}}
\end{equation}

In Equation (\ref{eq:wt}), \(w_t\) adopts a form resembling logistic
function. It indicates three properties of attention. First, the sum of
all \(w_t\) equals 1, which indicates the attention available to be
allocated across sequential outcomes is limited. Second, \(w_t\) is
increasing in \(x_t\), indicating the decision maker is naturally
inclined to allocate more attention to periods with larger potential
rewards. This is in line with existing evidence, suggesting people often
pay much attention to pleasant information and avoid to know unpleasant
information (e.g.~ostrich effect). Third, it appears that \(w_t\) is
anchored in \(d_t\). This indicates that \(d_t\) represents the original
attention weight assigned to each period, and the attention adjustment
process (converting \(d_t\) to \(w_t\)) is costly. The parameter
\(\lambda\) quantifies the cost of attention adjustment.

I provide two rationales for Equation (\ref{eq:wt}). The first is based
on rational inattention theories. The second is an axiomatic theory. I
present the first here and discuss the second in Section \ref{axiom}.

The necessary notations and definitions are described as follows. Let
\(S_T=[s_0,s_1,...,s_T]\) be a potential realization of \(X_T\), and
\(\mathcal{S}(X_T)\) be the support of \(X_T\), i.e.~the smallest set
containing any potentially realized sequence \(S_T\), where
\(\mathcal{S}(X_T)\subseteq \mathbb{R}_{\geq 0}^{T+1}\). In each
\(S_T\), the attention weight assigned to period \(t\) is decided by a
weight function \(w(s_t)\). I follow Noor and Takeoka
\citetext{\citeyear{noor_optimal_2022}; \citeyear{noor_constrained_2023}}
to define the process that generates \(w(s_t)\) as \emph{constrained
optimal discounting}. The key assumption of constrained optimal
discounting process is that, when evaluating a reward sequence, the
decision maker wants to find an allocation policy for attention weights,
in order to maximize the anticipatory utility that she can obtain from
the sequence. However, this attention re-allocation process triggers a
cognitive cost. She needs to balance the benefit of focusing on periods
containing larger rewards and the cognitive cost of shifting attention.
The formal definition is given by Definition 1.

\textbf{Definition 1}: Let \(W\) be the support of all possible weight
functions. Given a stochastic reward sequence \(X_T\), the following
optimization problem is called the \emph{constrained optimal
discounting} problem for \(X_T\):\[ 
\begin{aligned}
\max_{w\in W}  \quad & \sum_{S_T\in \mathcal{S}(X_T)}\sum_{t=0}^T w(s_t)u(s_t) - C(w) \\
s.t. \quad &  \sum_{S_T\in \mathcal{S}}\sum_{t=0}^T w(s_t)=m \\
& w(s_t)\geq 0, \forall t\in\{0,1,…,T\} \\
\end{aligned}
\]where \(m\) is a constant,
\(C:[0,m]^{T+1}\rightarrow \mathbb{R}_{>0}\) is called \emph{information
cost} function, \(\partial C/\partial w(s_t)>0\) and
\(\partial^2 C/\partial w(s_t)^2>0\). That is, the information cost is
increasing and convex in \(w(s_t)\).

In Definition 1, the cognitive cost of shifting attention is decided by
an information cost function \(C(.)\) . This implies that the cognitive
cost is generated by focusing on certain reward information. A
well-known specification of \(C(.)\) is the Shannon cost function,
proposed by \citet{matejka_rational_2015}. The Shannon cost function was
originally used to justify the multinominal logit model in discrete
choice analysis, and so far has been topical in rational inattention
literature. To construct this style of information cost function,
\citet{matejka_rational_2015} introduce three assumptions. The first is
that the sum of all weights \(m\equiv1\). The second assumption is,
before acquiring any information, the decision maker establishes an
initial weight allocation for different attributes (outcomes), which
remains invariant across states. The weights are then updated in a
manner consistent with Bayes rule. In ADU setting, it means
\(d_t=\sum_{S_T\in \mathcal{S}(X_T)} w(s_t)\). The third assumption is,
the information cost is linear to the information gains, measured by
Shannon mutual information. That
is,\[ C(w)= \lambda \sum_{S_T\in \mathcal{S}(X_T)}\sum_{t=0}^T w(s_t) \log\left(\frac{w_t(S_T)}{d_t}\right) \]where
\(\lambda\) is a parameter denoting unit cost of information
(\(\lambda>0\)). With this Shannon cost function, the constrained
optimal discounting problem can be easily solved by Lagrangian method,
and the solution is the same as Equation (\ref{eq:wt}).

\hypertarget{implications-in-intertemporal-choice}{%
\subsection{Implications in Intertemporal
Choice}\label{implications-in-intertemporal-choice}}

Here I document six implications of the ADUS model in intertemporal
choice. Each of them can be precisely proved.

\begin{enumerate}
\def\labelenumi{\arabic{enumi}.}
\item
  The model is consistent with ``hidden zero effect''.
\item
  The requirement of magnitude effect on curvature of utility function
  can be relaxed under ADUS.
\item
  The model specifies a novel condition for common difference effect.
\item
  The model suggests that the discount function may be concave for the
  near future and convex for the far future.
\item
  The model offers an alternative account for S-shaped value function.
\item
  In dynamic budget allocation, decision makers who exihibit greater
  patience at the beginning will perform less inconsistent behaviors in
  subsequent periods.
\end{enumerate}

\hypertarget{axiomatic-characterization}{%
\subsection{\texorpdfstring{Axiomatic Characterization
\label{axiom}}{Axiomatic Characterization }}\label{axiomatic-characterization}}

\hypertarget{the-role-of-attention-updating-in-inconsistent-planning}{%
\subsection{The Role of Attention Updating in Inconsistent
Planning}\label{the-role-of-attention-updating-in-inconsistent-planning}}

\hypertarget{empirical-analysis}{%
\subsection{Empirical Analysis}\label{empirical-analysis}}

\hypertarget{limited-and-motivated-attention-over-sequential-outcomes}{%
\section{Limited and Motivated Attention over Sequential
Outcomes}\label{limited-and-motivated-attention-over-sequential-outcomes}}

\hypertarget{valuation-of-risk-and-effort}{%
\section{Valuation of Risk and
Effort}\label{valuation-of-risk-and-effort}}

\hypertarget{time-schedule}{%
\section{Time Schedule}\label{time-schedule}}

\renewcommand\refname{Reply to Comments}
  \bibliography{reference.bib}

\end{document}
