
\documentclass[12pt]{article}


\begin{document}
\begin{table}
    \caption{Regression Results}
    \vspace*{12pt}
    \centering

      \begin{tabular}{lllllll}
\hline
 & (1) Pool & (2) Pool & (3) FE & (4) FE & (5) RLM & (6) RLM \\
\hline
$Y_1\cdot1\{T=T_L\}$ & -0.006$^{***}$ &  & -0.006$^{**}$ &  & -0.005$^{***}$ &  \\
 & (0.002) &  & (0.002) &  & (0.001) &  \\
$Y_1\cdot1\{T=T_H\}$ & -0.006$^{***}$ &  & -0.006$^{**}$ &  & -0.007$^{***}$ &  \\
 & (0.002) &  & (0.002) &  & (0.001) &  \\
$Y_1\cdot1\{T=T_L\}\times$CL1 &  & 0.023$^{***}$ &  & 0.002 &  & 0.0 \\
 &  & (0.004) &  & (0.002) &  & (0.001) \\
$Y_1\cdot1\{T=T_H\}\times$CL1 &  & 0.024$^{***}$ &  & 0.003 &  & -0.0 \\
 &  & (0.004) &  & (0.002) &  & (0.001) \\
$Y_1\cdot1\{T=T_L\}\times$CL2 &  & -0.06$^{***}$ &  & -0.02$^{***}$ &  & -0.018$^{***}$ \\
 &  & (0.005) &  & (0.004) &  & (0.002) \\
$Y_1\cdot1\{T=T_H\}\times$CL2 &  & -0.061$^{***}$ &  & -0.022$^{***}$ &  & -0.022$^{***}$ \\
 &  & (0.005) &  & (0.004) &  & (0.002) \\
PELI & -0.705 & -1.218 & 8.796$^{***}$ & 6.52$^{***}$ & 1.431$^{***}$ & 1.439$^{***}$ \\
 & (3.815) & (2.552) & (0.014) & (0.392) & (0.33) & (0.322) \\
Constant & 54.194$^{***}$ & 54.586$^{***}$ & 46.971$^{***}$ & 48.711$^{***}$ & 52.766$^{***}$ & 53.026$^{***}$ \\
 & (3.461) & (2.603) & (0.507) & (0.449) & (0.375) & (0.365) \\\hline

observations & 2242 & 2242 & 2242 & 2242 & 2254 & 2254 \\
adj-$R^2$ & 0.001 & 0.335 & 0.637 & 0.645 &  &  \\
Muller-Welsh &  &  &  &  & 146.734 & 139.589 \\
\hline
\end{tabular}
% INSERT reg_combined

    \vspace*{4pt}
    \centering
    \begin{minipage}{1.0\textwidth}
    {\par\footnotesize Note: * $p<0.05$, ** $p<0.01$, *** $p<0.005$. Standard errors are reported in the parentheses. Model (1)-(2) are pooled OLS models, model (3)-(4) are fixed-effect OLS models, model (5)-(6) are fixed-effect robust linear regressions (RLM). For OLS, standard errors are clustered at the subject level, and p-values are calculated using t-tests. For RLM, each model is estimated using Huber's M-estimator (the threshold for loss function is set at 1.345) and the scale estimator is Huber's proposal 2 estimator. Each p-value for RLM is calculated based on a normal distribution with i.i.d. assumption. A smaller Muller-Welsh score indicates the model has a greater ability to both parsimoniously fit the data and predict new independent obeservations. $Y_1$ and $T$ denote the front-end amount and the sequence length in Option A. $T_L$ and $T_H$ are 6 months and 12 months, respectively. Clustering results are obtained through k-means method.}
    \end{minipage}
    \label{tab:seq_value_reg}
\end{table}

\end{document}

